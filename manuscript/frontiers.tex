%%% Version 4.4 Generated 2018/06/15 %%%
\documentclass[utf8]{style/FrontiersinHarvard}
\usepackage{url, hyperref, lineno, microtype}
\usepackage[onehalfspacing]{setspace}

\usepackage{multirow}
\usepackage{xspace}
\usepackage{siunitx}
\usepackage[version=4]{mhchem}
\usepackage{textcomp} % registered trademark and copyright symbols
\usepackage{graphicx, subcaption}

% See mode, number-mode and unit-mode; values are text, math, and match
\sisetup{separate-uncertainty=true, 
    exponent-product=\cdot,
    per-mode=symbol,
    mode = match,
    range-units=single}
\DeclareSIUnit{\year}{yr}
\graphicspath{{"figures/"}}

% \linenumbers
\def\keyFont{\fontsize{8}{11}\helveticabold }

% Set author information
% --------------------------------------------- %
\def\firstAuthorLast{Mastnak {et~al.}}
\def\Authors{Mastnak, Elijan J.\,$^{1,*}$, Djordević, Srdjan\,$^{2}$ and Krašna, Simon\,$^{3}$}
\def\Address{$^{1}$University of Ljubljana, Faculty of Mathematics and Physics, Ljubljana, Slovenia. elijan-jakob.mastnak@student.fmf.uni-lj.si \\
$^{2}$TMG-BMC Ltd., Štihova ulica 24, 1000 Ljubljana, Slovenia. srdjand@tmg.si \\
$^{3}$University of Ljubljana, Faculty of Mechanical Engineering, Ljubljana, Slovenia. simon.krasna@fs.uni-lj.si}
\def\corrAuthor{Elijan Mastnak}
\def\corrEmail{elijan-jakob.mastnak@student.fmf.uni-lj.si}

% --------------------------------------------- %

% Define custom macros
% --------------------------------------------- %
\newcommand{\TODO}[1]{{\textbf{TODO:} {\color{red} #1}}}

% TMG parameters
% --------------------------------------------- %
\newcommand{\Dm}{\ensuremath{\text{Dm}}\xspace}
\newcommand{\Td}{\ensuremath{\text{Td}}\xspace}
\newcommand{\Tc}{\ensuremath{\text{Tc}}\xspace}
\newcommand{\RDDMax}{\ensuremath{ \text{RDD}_{\text{max}}}\xspace}
\newcommand{\RDDMaxTime}{\ensuremath{ \text{TRDD}_{\text{max}}}\xspace}
% --------------------------------------------- %

% SPM parameters
% --------------------------------------------- %
\newcommand{\SPMStart}{\ensuremath{\text{Dm}}\xspace}
\newcommand{\SPMEnd}{\ensuremath{\text{Td}}\xspace}
\newcommand{\SPMCentroidTime}{\ensuremath{\text{Tc}}\xspace}
\newcommand{\SPMCentroidT}{\ensuremath{\text{Dm}}\xspace}
\newcommand{\SPMMax}{\ensuremath{\text{Dm}}\xspace}
\newcommand{\SPMArea}{\ensuremath{\text{Dm}}\xspace}
% --------------------------------------------- %

\begin{document}
\onecolumn
\firstpage{1}

\title[Quantifying Twitch Potentiation with SPM]{Detecting and Quantifying Post-Activation Twitch Potentiation with Statistical Parametric Mapping} 

\author[\firstAuthorLast ]{\Authors}
\address{}
\correspondance{}
\extraAuth{}

\maketitle

% The abstract should render the general significance and conceptual advance of the work clearly accessible to a broad readership.
\begin{abstract}
\section{}
This study proposes and demonstrates the use of statistical parametric mapping for reliably detecting and quantifying post-activation twitch potentiation in superficial skeletal muscles.
The study applies SPM analysis to tensiomyographic measurements of muscle displacement with respect to time in electrically-induced twitch contractions immediately following short bursts of intense muscular activity.

\tiny
\keyFont{\section{Keywords:} tensiomyography, statistical parameteric mapping, twitch potentiation, post-activation potentiation, neuromuscular electrical stimulation}
% Provide 5-8 keywords
\end{abstract}

\section{Introduction}
In the phenomenon generally called post-activation potentiation, a short, intense burst of muscle activity (called a \textit{conditioning exercise} in this context) temporarily increases the amplitude, force, and speed of both voluntary and electrically induced muscle contractions that immediately follow.
[14][15][1][16][17].

In general, prior muscle activity produces two different---and potentially coexistent---muscle responses: fatigue and potentiation.
Limited by the inevitable onset of fatigue, one cannot use conditioning exercises to increase subsequent muscle performance via potentiation ad infinitum---eventually fatigue outweighs any potentiation-like improvement in muscle performance.
Fatigue can coexist with PAP [18], and measured muscular performance represents the net balance between processes that cause fatigue and processes that cause potentiation [18].
Striking a favorable balance between fatigue and potentiation should be of interest to coaches, sports scientists, and athletes alike.
Doing so requires a reliable means of detecting and quantifying potentiation, and this article offers a novel way of doing so---directly in the time domain in which muscle response is measured.

\subsection{On PAP, PAPE, and twitch potentiation}
We pause briefly to comment on some ambiguity in the literature surrounding post-conditioning enhancements in muscular performance,
generally involving confusion between the following two phenomena.
% \cite{prieske}
\begin{itemize}

    \item \textit{Post-activation potentiation} (PAP) is a short-term enhancement in a muscle's contractile properties during an electrically-induced muscle twitch, which generally manifests as enhanced twitch force and amplitude, and decreased time to peak contraction.
    PAP has a short half-life ($ \sim \SI{30}{\second} $),
    % \cite{vandervoort}
    is verified at the specific level of an individual muscle using a non-voluntary, electrically-induced muscle twitch, and generally requires nontrivial equipment and a standardized test environment to measure.
% study in a controlled, systematic manner.

    \item \textit{Post-activation performance enhancement} (PAPE) is
    a somewhat longer-term performance improvement in exercises serving as measures of maximal strength, power, and speed (e.g. a maximum vertical jump or a 60-meter sprint).
    PAPE has a longer half-life (maximum performance enhancement occurs on the order of 5 to 10 minutes following the initial conditioning exercise),
    % \cite{wilson}
    is verified at a macroscopic level by observing exercise performance (rather than measuring the contractile properties of a specific muscle),
    and is thus simpler to measure than than PAP, possibly at the expense of a less controlled and systematic experiment.

\end{itemize}

The present study concerns PAP, i.e. the short-term, post-conditioning enhancement in muscular contractile properties during electrically-induced twitch, and for this reason we will tend to use the more specific term \textit{twitch potentiation}.
Borrowing the definition of Hodgson et al (2005) [2],
``a twitch is a brief muscle contraction in response to a single presynaptic action potential or a single, synchronised volley of action potentials.''
Twitch potentiation (TP) is a well-established and reproducible phenomenon that manifests as both enhanced muscle twitch amplitude [19][27][28] and decreased time to attaining maximum amplitude [4] \cite{sale} [21].

TP has been demonstrated using electrically induced twitch contractions and attributed to phosphorylation of myosin regulatory light chains [21], which makes actin and myosin more sensitive to \ce{Ca^2+} ions [2]
The muscle's potentiated state has also been attributed to an increase in $ \alpha $-motoneuron excitability [16].
Fatigue induces a decrease in amplitude and velocity of twitch contraction, while PAP causes an increase in amplitude and velocity of twitch contraction and thus enhanced contractile response [4][2][3].

\subsection{Quantifying potentiation}
Currently, it is standard to quantify the level of twitch potentiation using discrete twitch contraction parameters computed retrospectively from measurements of muscle displacement, force production, or electrical activity with respect to time.
Examples of these parameters include maximal amplitude of muscle contraction, time taken to reach peak contraction, maximal twitch rate of force development [29][30][31], and M-wave amplitude [3].

In each case, computing these parameters reduces the originally-measured muscle response, which is generally a one-dimensional time series (e.g. muscle displacement with respect to time), to a single number, i.e. a zero-dimensional scalar value.
Based on the values of these parameters, one then retrospectively infers if the muscle response from which they were computed corresponded to a potentiated state.

A promising new approach to quantifying TP would analyze 
muscle response and detect and quantify potentiation directly in the time domain in which the muscle response was originally measured, without the dimensionality reduction imposed by computing discrete twitch contraction parameters.
One way of performing such an analysis is the method of statistical parametric mapping (SPM) [32][33][34][35].

Borrowing the definition of Flandin and Friston \cite{flandin}, 
``Statistical parametric mapping is the application of random field theory to make inferences about the topological features of statistical processes that are continuous functions of space or time''.
SPM has so far found its greatest application in neuroimaging for detecting regionally-specific brain activations \cite{friston}.
We use SPM similarly in the present study---applied to one-dimensional functions of time (i.e. twitch amplitude with respect to time) instead of three-dimensional functions of space (i.e. brain scans)---to detect regionally-specific variations between pre-conditioning and post-conditioning muscle responses directly in the time domain.
Statistically significant variations between pre-conditioning and post-conditioning muscle responses may then be interpreted as post-activation potentiation.
Most promisingly, because SPM preseves the time-domain information contained in the original muscle response, SPM shows exactly \textit{when} the potentiation-like state occurred over the course of the muscle twitch.
Of course SPM and conventional twitch contraction parameters are far from mutually exclusive as a means of detecting and quantifying potentiation---in fact the two may be used simultaneously, as the present study illustrates.

Quantifying twitch potentiation, then, requires two things: a means of measuring a muscle's contractile response following a conditioning exercise and a technique for analyzing the measured muscle response that can determine the presence (or lack) of potentiation.
Tensiomyography (TMG) is a standard method for selectively measuring contractile response and assessing twitch contraction parameters in superficial skeletal muscles [36][37][38][39][40][41][42].
This study assessed TP in a sports training setting by analyzing TMG measurements of electrically-induced muscle contraction both
\begin{enumerate}

    \item using SPM applied directly to TMG muscle responses in the time domain, and

    \item using discrete (zero-dimensional) TP parameters, computed retrospectively from the measured TMG signal, that summarize the muscle's twitch contraction properties.

\end{enumerate}
The study aimed to find if TMG measurement of muscle response followed by SPM analysis of this response can accurately and consistently identify statistically significant differences in the pre- and post-conditioning contractile response of the rectus femoris muscle, and,
if yes, to identify and quantify the associated time-domain changes of twitch potentiation.

\section{Materials and Methods}
\subsection{Participants}
The present study analyzed 55 subjects (age \SI{28.17 \pm 10.9}{\year}; height \SI{178.15 \pm 10.43}{\centi \meter}; weight \SI{71.84 \pm 12.82}{\kilogram}, 28 male and 27 female) using an approximation of a simple random sampling strategy.
% TODO describe the simple random random sampling strategy?
To be approved for participation, subjects were required to be injury-free at the beginning of the testing protocol.
The intention and experimental procedure was described to all subjects, and written informed consent was obtained.
The investigation was approved by the Institutional Research Ethics Committee (University of Belgrade, Faculty of Sport and Physical Education), following the recommendations of the Declaration of Helsinki.

\subsection{Study Design}
The study used tensiomyograhy (TMG) to measure pre- and post-ISQ changes in the twitch-contraction properties of the rectus femoris (RF) muscle.
The tensiomyographic measurement used an opto-inductive sensor with a spring of spring constant \SI{0.17}{\newton \per \milli \meter}, which generated an initial pressure of approximately \SI{1.5e-2}{\newton \per \milli \meter \squared} on a tip area of \SI{11.34}{\milli \meter \squared}.
Two self-adhesive electrodes (UltraStim\textregistered{} Wire, Axelgaard, USA) were positioned approximately along the anatomical axis of the femur, proximal on the beginning of RF and distal \SI{10}{\centi \meter} apart [11], symmetrically \SI{2}{\centi \meter} from the sensor on the RF muscle.
The positive electrode (anode) was placed proximally, and the negative electrode (cathode) distally, as shown in Figure 1.
% TODO: image of sensor position

The electrodes remained fixed on the skin for the duration of the measurement and exercise procedure.
The sensor tip's position on the rectus femoris muscle was marked for reference at the beginning of the measurement protocol, and the same tip position was used for all measurements.
A single-twitch electrical stimulus (a DC pulse of \SI{1}{\milli \second} duration) induced an isometric muscle contraction,
and the muscle response was recorded and analyzed using a standardized algorithm for the TMG S1 system (TMG-BMC, Ljubljana, Slovenia).
Three parameters were used to evaluate the TMG signal:
Td (delay time), Tc (contraction time), and Dm (maximum displacement of muscle belly during contraction).
A typical TMG signal, with all parameters defined, is shown in Figure 2.
% TODO: figure

The electrical stimulation intensity applied to the rectus femoris muscle was increased using the standard TMG protocol [36][37][38].
After each twitch electrical stimulus, the electrical current was increased until the muscle reached supramaximal response;
the current level for supramaximal muscle response was then used for electrical stimulation in all subsequent measurements.

The study used the following protocol of exercises and measurements:
% TODO: describe protocol
We used the following potentiation protocol of exercises and measurements, repeated four times in a precise time sequence:
\begin{itemize}

    \item pre-ISQ measurement (PRE), followed within \SI{15}{\second} by

    \item incline squat, followed within \SI{12}{\second} by

    \item post-ISQ measurement (POE), followed by

    \item a rest period of \SI{150}{\second}.

\end{itemize}
The above described procedure was repeated four times.
The complete procedure and timing of the exercise measurement protocol are depicted in Figure 3.
% TODO reference figure 3

The volunteers performed ISQ on a platform with a slope angle of $ \ang{30} $, holding an additional load of between $ \SI[parse-numbers = false]{2 \times 0}{\kilogram} $ and $ \SI[parse-numbers = false]{2 \times 20}{\kilogram} $ in each hand (Figure 4a, 4b);
% TODO figure 4a and 4b
the load was chosen based on a ten-repetition maximum test performed the day before.
The time of the squat movement was \SI{1}{\second} down and \SI{1}{\second} up (a metronome kept time).
The knee angle ranged between $ \ang{0} $ and $ \ang{90} $, while the torso-femur angle remained constant at $ \ang{0} $ throughout the movement.
Each subject performed four sets of ISQ, with eight repetitions in each set and a \SI{150}{\second} rest between sets.

The effect of the incline squat on the rectus femoris muscle was measured with TMG within \SI{12}{\second} of the final repetition in each set.
A pre-ISQ measurement preceded the second, third and fourth sets.
Subjects began each set of ISQ with \SI{15}{\second} after the pre-ISQ measurement.

An inverse dynamics analysis of a three-segment planar human body model, assuming a quasi-static condition (see Appendix 1) [10][7][8], was used to estimate the load on the quadriceps muscles (especially the RF, which is loaded at the full amplitude of knee movement) during the ISQ.
The model's body segment properties were estimated using anthropometric data and the subject’s body height and body mass as input parameters [9].
% TODO: clarify with the anthropometric data is. Is this the distance from knee to foot?
The subjects' motion in the sagittal plane was recorded on video, and one exercise repetition was selected for estimating the inter-segmental angles (Figure 4).
The peak value of the net knee joint moment was estimated at the lowest body position in the ISQ,
and normalized to the volunteer’s body mass to reduce the effect of inter-subject variance [5][6].
The normalized peak values of the knee joint moment were used to compare the loading of the knee extensor muscles during the ISQ between the volunteers.

\subsection{Statistical Analyses}
Standard statistical tests were made using SPSS Statistics 20 (SPSS Inc., Chicago, IL, USA) and Microsoft Excel.
Statistical parametric mapping analysis of TMG signals was performed in the Python 3 programming language (\url{https://www.python.org/})
% TODO reference Python  
using the SPM1d library [34][35] (\url{https://www.spm1d.org}).

% Statistical parametric mapping applies random field theory to make inferences about the topological features of statistical processes that are continuous functions of space or time. 
% TODO describe coherently what SPM does

Descriptive statistics were calculated on each twitch potentiation and SPM parameter, and the Shapiro–Wilk test was used to verify the normality of the data.
Data are presented as means $ \pm $ standard deviations (SD).
The differences between pre-ISQ and post-ISQ conditions were determined using Student’s dependent $ t $-test for paired samples.
All statistical tests were performed at the significance level $ \alpha = 0.05 $.
% A few times we have had "significance set to p < 0.5.
% What does that mean?

% TODO make this entire description of SPM coherent
One dimensional paired SPM $ t $-tests were used to determine the main effects of ISQ on the signal and twitch-contraction properties of the RF muscle.
The $ p $-value was calculated for clusters crossing the critical threshold ($ t^{*} $), with the significance level set to $ \alpha = 0.05 $ (Pataky, 2013 [13]).
The significance level $ \alpha = 0.05 $ was set for all analyses, and specific p-values were reported for each supra-threshold region of the SPM t-continuum.
The first \SI{100}{\milli \second} of the RF twitch contraction signal was used as the region of interest for all SPM analyses.
One-dimensional SPM analysis returns regions of significance in the form of clusters (Pataky).
These are contiguous values over which the curve is determined to be not consistent with random sampling.
The following variables of the SPM statistics were collected or computed:
\begin{itemize}

    \item diff-set is the difference between pre- and post-ISQ (ISQ) twitch contraction measured on rectus femoris muscle;

    \item Centroid time is the time of the centroid of area Above threshold (see Figure 5a-d);

    \item Centroid t-value is the position of the SPM t-axis value of area Above Threshold centroid;

    \item Maximum is the t maximum value;

    \item Area Above Threshold is the area (integral) of statistically significant difference between the mean of pre- and post-ISQ twitch contraction applying SPM t-statistical analysis and corresponds to the level of potentiation;

    \item Start Time is beginning of the statistically significant difference between pre- and post-ISQ mean of rectus femoris twitch contraction;

    \item End Time is the end of the statistically significant difference between pre and post-ISQ mean of rectus femoris twitch contraction.

\end{itemize}

\section{Results}
In all four measurement sets, the difference in the volunteers' pre-ISQ and post-ISQ twitch contraction parameters 
(\Td, \Tc, \Dm, \RDDMax, and \RDDMaxTime)
was statistically significant, with a $ p $ value $ p < 0.0001 $ when tested with a dependent Student's $ t $-test for paired samples at a significance level $ \alpha = 0.05 $.
The study used two related schemes to compare pre- and post-ISQ twitch contraction parameters:
\begin{enumerate}

    \item In the first method, pre- and post-ISQ twitch contraction parameters are compared set by set;
    the parameter values appear in Table~\ref{tab:tmg_params}.

    \item In the second method, post-ISQ twitch parameters in each set are compared to the pre-ISQ parameters in the \textit{first} set;
    the parameter values appear in Table~\ref{tab:tmg_params_staggered}.
    % TODO: reasoning why this method is used.

\end{enumerate}

\begin{table}[htb!]
    \centering
    \caption{Set-by-set comparison of pre- and post-ISQ twitch contraction parameter values averaged across all subjects---note the consistent, potentiation-like increase in muscle amplitude and decrease in contraction time following ISQ.
    The difference between pre- and post-ISQ values for all parameters was statistically with a $ p $ value $ p < 0.0001 $ when tested with a Student's $ t $-test for paired samples at a significance level $ \alpha = 0.05 $.
    Parameters were introduced in Materials and Methods; for review:
    \Dm is maximum displacement of TMG sensor from muscle belly;
    \Td is time from start of TMG signal to 10\% of its maximum value \Dm;
    \Tc is time from 10\% of \Dm to 90\% of \Dm;
    \RDDMax is maximum value of the TMG signal's time derivative,
    \RDDMaxTime is time at which \RDDMax occurs.
    }
    \vspace{1ex}

    \renewcommand{\arraystretch}{1.2}
    \begin{tabular}{|c|l|c|c|c|c|}
    \hline {\rule{0pt}{2.0ex}} \hspace{-7pt}
    Parameter & & Set 1 & Set 2 & Set 3 & Set 4\\
    \hline
    \hline {\rule{0pt}{2.0ex}} \hspace{-7pt}
    
    % Dm
    \multirow{3}{*}{\textbf{Dm}} & Mean PR $ [\si{\milli \meter}] $ & $8.71$ & $9.21$ & $9.07$ & $9.13$\\
     & Mean PO $ [\si{\milli \meter}] $ & $10.06$ & $10.10$ & $10.15$ & $9.95$\\
     & Percent difference & $+15.42$\% & $+9.70$\% & $+11.83$\% & $+8.96$\%\\
    \hline {\rule{0pt}{2.0ex}} \hspace{-7pt}
    
    % Td
    \multirow{3}{*}{\textbf{Td}} & Mean PR $ [\si{\milli \second}] $ & $25.28$ & $24.97$ & $24.01$ & $24.39$\\
     & Mean PO $ [\si{\milli \second}] $ & $22.46$ & $22.15$ & $21.94$ & $21.85$\\
     & Percent difference & $-11.17$\% & $-11.28$\% & $-8.64$\% & $-10.41$\%\\
    \hline {\rule{0pt}{2.0ex}} \hspace{-7pt}
    
    % Tc
    \multirow{3}{*}{\textbf{Tc}} & Mean PR $ [\si{\milli \second}] $ & $31.38$ & $30.59$ & $29.36$ & $28.94$\\
     & Mean PO $ [\si{\milli \second}] $ & $26.17$ & $25.24$ & $24.92$ & $24.56$\\
     & Percent difference & $-16.61$\% & $-17.51$\% & $-15.12$\% & $-15.12$\%\\
    \hline {\rule{0pt}{2.0ex}} \hspace{-7pt}
    
    % RDD max
    \multirow{3}{*}{\textbf{$ \text{RDD}_{\text{max}} $}} & Mean PR $ [\si{\milli \meter \per \milli \second}] $ & $0.27$ & $0.29$ & $0.30$ & $0.31$\\
     & Mean PO $ [\si{\milli \meter \per \milli \second}] $ & $0.37$ & $0.39$ & $0.40$ & $0.39$\\
     & Percent difference & $+39.52$\% & $+35.10$\% & $+31.25$\% & $+27.46$\%\\
    \hline {\rule{0pt}{2.0ex}} \hspace{-7pt}
    
    % RDD max time
    \multirow{3}{*}{\textbf{$ \text{TRDD}_{\text{max}} $}} & Mean PR $ [\si{\milli \second}] $ & $41.03$ & $39.66$ & $38.72$ & $38.55$\\
     & Mean PO $ [\si{\milli \second}] $ & $36.56$ & $35.28$ & $35.15$ & $35.30$\\
     & Percent difference & $-10.90$\% & $-11.04$\% & $-9.24$\% & $-8.43$\%\\
    \hline
\end{tabular}

    \label{tab:tmg_params}
\end{table}

\begin{table}[htb!]
    \centering
    \caption{Post-ISQ twitch contraction parameter values from sets 1, 2, 3, and 4 compared to the pre-ISQ values from set 1,
    so mean pre-ISQ values are shown only for set 1.
    All parameters have the same meanings as in Table~\ref{tab:tmg_params}.
    As in Table~\ref{tab:tmg_params}, note the consistent, potentiation-like increase in muscle amplitude and decrease in contraction time.
    }
    \vspace{1ex}

    \renewcommand{\arraystretch}{1.2}
    \begin{tabular}{|c|l|c|c|c|c|}
    \hline {\rule{0pt}{2.0ex}} \hspace{-7pt}
    Parameter & & Set 1 & Set 2 & Set 3 & Set 4\\
    \hline
    \hline {\rule{0pt}{2.0ex}} \hspace{-7pt}
    
    % Dm
    \multirow{3}{*}{\textbf{Dm}} & Mean PR $ [\si{\milli \meter}] $ & $8.71$ & - & - & -\\
     & Mean PO $ [\si{\milli \meter}] $ & $10.06$ & $10.10$ & $10.15$ & $9.95$\\
     & Percent difference & $+15.42$\% & $+15.95$\% & $+16.44$\% & $+14.17$\%\\
    \hline {\rule{0pt}{2.0ex}} \hspace{-7pt}
    
    % Td
    \multirow{3}{*}{\textbf{Td}} & Mean PR $ [\si{\milli \second}] $ & $25.28$ & - & - & -\\
     & Mean PO $ [\si{\milli \second}] $ & $22.46$ & $22.15$ & $21.94$ & $21.85$\\
     & Percent difference & $-11.17$\% & $-12.38$\% & $-13.22$\% & $-13.58$\%\\
    \hline {\rule{0pt}{2.0ex}} \hspace{-7pt}
    
    % Tc
    \multirow{3}{*}{\textbf{Tc}} & Mean PR $ [\si{\milli \second}] $ & $31.38$ & - & - & -\\
     & Mean PO $ [\si{\milli \second}] $ & $26.17$ & $25.24$ & $24.92$ & $24.56$\\
     & Percent difference & $-16.61$\% & $-19.59$\% & $-20.60$\% & $-21.73$\%\\
    \hline {\rule{0pt}{2.0ex}} \hspace{-7pt}
    
    % RDD max
    \multirow{3}{*}{\textbf{$ \text{RDD}_{\text{max}} $}} & Mean PR $ [\si{\milli \meter \per \milli \second}] $ & $0.27$ & - & - & -\\
     & Mean PO $ [\si{\milli \meter \per \milli \second}] $ & $0.37$ & $0.39$ & $0.40$ & $0.39$\\
     & Percent difference & $+39.52$\% & $+46.89$\% & $+48.16$\% & $+45.62$\%\\
    \hline {\rule{0pt}{2.0ex}} \hspace{-7pt}
    
    % RDD max time
    \multirow{3}{*}{\textbf{$ \text{TRDD}_{\text{max}} $}} & Mean PR $ [\si{\milli \second}] $ & $41.03$ & - & - & -\\
     & Mean PO $ [\si{\milli \second}] $ & $36.56$ & $35.28$ & $35.15$ & $35.30$\\
     & Percent difference & $-10.90$\% & $-14.02$\% & $-14.35$\% & $-13.97$\%\\
    \hline
\end{tabular}

    \label{tab:tmg_params_staggered}
\end{table}

SPM anaysis of TMG measurements using a two-sample SPM $ t $-test showed a statistically significant difference in pre-ISQ and post-ISQ muscle twitch in all four sets, in each case with a p-value $ p < 0.0001 $ at a significance level $ \alpha = 0.05 $.
The first \SI{100}{\milli \second} of the TMG signal were used as the region of interest for all SPM analyses, because this is the region in which both the maximal amplitude of twitch contraction and the phase of twitch amplitude decrease occur.
The results of each set's SPM $ t $-tests are plotted in Figure~\ref{fig:spm_plot}, while discrete parameters summarizing each set's SPM test appear in Table~\ref{tab:spm_params}.
% The value of the SPM $ t $-statistic significance threshold $ t^{*} $ remained approximately constant across all four measurement sets, ranging between $ t^{*} = 2.76 $ and $ t^{*} = 2.79 $.

\begin{figure}
	\centering
    \includegraphics[width=\textwidth]{spm-plot.jpg}
    \caption{For all four sets, a two-sample SPM $ t $-test comparing the first \SI{100}{\milli \second} of normalized pre- and post-ISQ TMG measurements across all subjects found a statistically significant difference between pre- and post-ISQ muscle twitch responses.
    The left column shows mean normalized pre-ISQ (black solid line) and post-ISQ (red solid line) TMG measurements, together with their standard deviation clouds, across all subjects for each measurement set.
    The right column shows the results of the SPM $ t $-test for each set;
    the dashed line denotes the threshold $ t^{*} $ at which each set's pre- and post-ISQ TMG signals differ significantly, while the maroon cloud emphasizes the region of significance.}
    \label{fig:spm_plot}
\end{figure}

\begin{table}[htb!]
    \centering
    \caption{
        Discrete parameters summarizing the results of the two-sample SPM $ t $-tests show in Figure~\ref{fig:spm_plot}.
        Parameters were introduced in Materials and Methods; for review:
        ``Start time'' is the time at which the SPM $ t $-statistic attains the significance threshold value $ t^{*} $;
        ``End time'' is the time at which the SPM $ t $-statistic falls below the significance threshold value $ t^{*} $;
        ``Centroid time'' and ``Centroid $ t $-value'' give the (time, $ t $) coordinate of the SPM significance region's centroid (the region shown in maroon in Figure~\ref{fig:spm_plot});
        ``SPM maximum'' is maximum value attained by the SPM $ t $-statistic;
        ``Area above threshold'' is area of the SPM significance region.
    }
    \vspace{1ex}

    \renewcommand{\arraystretch}{1.1}
    \begin{tabular}{|l|c|c|c|c|}
    \hline {\rule{0pt}{2.0ex}} \hspace{-7pt}
    SPM Parameters & Set 1 & Set 2 & Set 3 & Set 4\\
    \hline
    \hline {\rule{0pt}{2.0ex}} \hspace{-7pt}
    Start time $ [\si{\milli \second}] $ & 10.37 & 10.63 & 10.07 & 9.87\\[0.3ex]
    End time $ [\si{\milli \second}] $ & 57.82 & 57.67 & 56.11 & 56.84\\[0.3ex]
    Centroid time $ [\si{\milli \second}] $ & 33.32 & 34.14 & 32.17 & 32.56\\[0.3ex]
    Centroid $ t $-value & 6.78 & 8.08 & 6.90 & 7.06\\[0.3ex]
    SPM maximum & 7.77 & 9.49 & 8.28 & 8.12\\[0.3ex]
    Area above threshold & 195.1 & 259.5 & 198.5 & 209.9\\[0.3ex]
    \hline
\end{tabular}

    \label{tab:spm_params}
\end{table}

As an estimate of muscle load, the calculated mean value of peak knee torque during the ISQ was \SI{126.5 \pm 17.6}{\newton \meter}.

\section{Discussion}

\section*{Conflict of Interest Statement}
The authors declare that the research was conducted in the absence of any commercial or financial relationships that could be construed as a potential conflict of interest.

\section*{Author Contributions}
The Author Contributions statement must describe the contributions of individual authors referred to by their initials and, in doing so, all authors agree to be accountable for the content of the work.
Please see \href{https://www.frontiersin.org/about/policies-and-publication-ethics#AuthorshipAuthorResponsibilities}{here} for full authorship criteria.

\section*{Funding}
Details of all funding sources should be provided, including grant numbers if applicable.
Please ensure to add all necessary funding information, as after publication this is no longer possible.

\section*{Acknowledgments}
This is a short text to acknowledge the contributions of specific colleagues, institutions, or agencies that aided the efforts of the authors.

\section*{Supplemental Data}
 \href{http://home.frontiersin.org/about/author-guidelines#SupplementaryMaterial}{Supplementary Material} should be uploaded separately on submission, if there are Supplementary Figures, please include the caption in the same file as the figure. LaTeX Supplementary Material templates can be found in the Frontiers LaTeX folder.

\bibliographystyle{Frontiers-Harvard}
\bibliography{test}
%%% Please see the test.bib file for some examples of references

\end{document}
