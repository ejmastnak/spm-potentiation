%%% Version 4.4 Generated 2018/06/15 %%%
\documentclass[utf8]{style/FrontiersinHarvard}
\usepackage{url, hyperref, lineno, microtype, subcaption}
\usepackage[onehalfspacing]{setspace}

\usepackage{graphicx}
\usepackage{mhchem}
\graphicspath{{"figures/"}}

% \linenumbers
\def\keyFont{\fontsize{8}{11}\helveticabold }

% Set author information
% --------------------------------------------- %
\def\firstAuthorLast{Sample {et~al.}}
\def\Authors{First Author\,$^{1,*}$, Co-Author\,$^{2}$ and Co-Author\,$^{1,2}$}
\def\Address{$^{1}$Laboratory X, Institute X, Department X, Organization X, City X , State XX (only USA, Canada and Australia), Country X \\
$^{2}$Laboratory X, Institute X, Department X, Organization X, City X , State XX (only USA, Canada and Australia), Country X  }
\def\corrAuthor{Corresponding Author}
\def\corrEmail{email@uni.edu}
% --------------------------------------------- %

% Define custom macros
% --------------------------------------------- %
\newcommand{\TODO}[1]{{\textbf{TODO:} {\color{red} #1}}}

\begin{document}
\onecolumn
\firstpage{1}

\title[Running Title]{Article Title} 

\author[\firstAuthorLast ]{\Authors}
\address{}
\correspondance{}
\extraAuth{}

\maketitle

\begin{abstract}
\section{}
As a primary goal, the abstract should render the general significance and conceptual advance of the work clearly accessible to a broad readership. References should not be cited in the abstract. Leave the Abstract empty if your article does not require one, please see \href{http://www.frontiersin.org/about/AuthorGuidelines#SummaryTable}{Summary Table} for details according to article type. 


\tiny
 \keyFont{ \section{Keywords:} keyword, keyword, keyword, keyword, keyword, keyword, keyword, keyword} % All article types: you may provide up to 8 keywords; at least 5 are mandatory.
\end{abstract}

\section{Introduction}
When muscle activity reaches a certain level of intensity and duration, it affects the subsequent performance of the muscles themselves.
More precisely, it affects the subsequent muscle load and contraction that can be performed voluntarily (task-dependent muscle activity) or by electrically induced contraction (twitch contraction, intermittent and tetanic) [14][15][1][16][17].
Repeated muscle activity can produce two completely different muscle responses: (1) fatigue and (2) potentiation or post-activation potentiation (PAP). 
Before proceeding, we first define a \textit{twitch}:
according to Hodgson et al (2005) [2], ``a twitch is a brief muscle contraction in response to a single presynaptic action potential or a single, synchronised volley of action potentials.''
At the muscle contraction level, fatigue induces a decrease in amplitude and velocity of twitch contraction, while PAP causes an increase in amplitude and velocity of twitch contraction and thus enhanced contractile response [4][2][3].
However, fatigue can coexist with PAP [18], and measured muscular performance represents the net balance between processes that cause fatigue and processes that cause potentiation [18].
PAP has been demonstrated using electrically induced twitch contractions and attributed to phosphorylation of myosin regulatory light chains [21], which makes actin and myosin more sensitive to \ce{Ca^2+} [2];
the muscle's potentiated state has also been attributed to an increase in $ \alpha $-motoneuron excitability [16].
The two most prevalent measures of neuromuscular output used to quantify the effect of the above-described muscle activation have been (1) muscle twitch amplitude [20], and (2) H-reflex and M-wave amplitude [3].

% TODO: define twitch potentiation
The analysis of twitch contraction induced by electrical stimulation following muscle activity is the standard method for determining twitch potentiation (TP).
TP is a well-established and reproducible phenomenon that manifests as both enhanced twitch amplitude [19][27][28] and decreased time to attaining maximum amplitude [4][20][21].
TP increases the force and rate of force development (RFD) in low-frequency tetanic isometric contractions, but increases RFD only in high frequency tetanic contractions.
Tensiomyography (TMG) is a standard method for experimental assessment of twitch contraction parameters on selective muscles [36][37][38][39][40][41][42].

The significance of PAP to functional performance has not been well established [2].
More research is required to clarify the functional significance of PAP and, in particular, the efficiency of training in producing long-term neuromuscular adaptations [2].
Scientists, sports coaches, and athletes can benefit from understanding potentiation and fatigue individually, and also the simultaneous occurrence of both phenomena. % TODO phrase better
A current interest in PAP studies related to sport and rehabilitation is finding specific muscle loading protocols that induce significant PAP, while minimizing fatigue and the risk of injury or repeated injury.
Developing simpler methods but maintaining comparable or improved reliability can be helpful for further understanding and quantifying TP and PAP. % TODO what does this mean?
The simplicity and specificity of potentiation-based measurement allows more reliable analysis of different types of motor tasks, exercise or rehabilitation strategies.
In particular, it would be helpful to quantify the level of twitch potentiation not only using discrete twitch contraction parameters (maximal amplitude, contraction time, maximal twitch RFD [29][30][31], M-wave amplitude [3]), but by analyzing and quantifying changes in the continuous time series, such as by applying the statistical parametric mapping (SPM) method [32][33][34][35].

This study aimed to assess TP in sports training diagnostics (mainly for speed and explosive power development) by analyzing TMG signals both (1) as continuous time series using SPM in the time domain and (2) using discrete TP parameters summarizing a muscle's twitch contraction properties.
Specifically, the study aimed to find if TMG and SPM can identify statistically significant differences between twitch contraction of the rectus femoris muscle in pre- and post-exercise states and, if yes, identify and quantify time-domain changes of twitch potentiation.

\section{Materials and Methods}
\subsection{Participants}
The present study analyzed 55 subjects (28.17±10.9 yr; 178.15±10.43 cm; 71.84±12.82 kg, 28male, and 27female) using an approximation of a simple random sampling strategy.
To be approved for participation in the research, subjects were required to be injury-free at the beginning of the testing protocol.
The intention and experimental procedure were described to all subjects, and written informed consent was obtained.
The current investigation was approved by the Institutional Research Ethics Committee (The University of Belgrade, The Faculty of sport and physical education), following the recommendations of the Declaration of Helsinki.

\subsection{Study Design}
For measuring the changes in twitch-contraction properties of rectus femoris muscle and twitch-potentiation like effect was used Tensiomypgrahy method.
For tensiomyographic measurements, an opto-inductive sensor including spring with the spring constant of 0.17 N/mm was used, which generates an initial pressure of approximately 1.5x10-2 N/mm2 on a tip area of 11.34 mm2.
Two self-adhesive electrodes (UltraStim® Wire, Axelgaard, USA) were positioned approximately in the anatomical axis of the femur, proximal on the beginning of RF and distal 10 cm apart [11], symmetrically 2 cm from the sensor on the rectus femoris muscle.
The positive electrode (anode) was placed proximally, and the negative electrode (cathode) distally as it shown in Figure \TODO{reference}.
The electrodes remained fixed on the skin during the complete measurement and exercise procedure.
The sensor's tip position was marked (for all measurements) at the beginning of the rectus femoris muscle.
A single-twitch electrical stimulus (a DC pulse of 1ms duration) induced an isometric muscle contraction.
Measured muscle responses were stored and analyzed using a standardized algorithm for the TMG S1 system (TMG-BMC, Ljubljana, Slovenia) to clarify the muscle response.

Three parameters to evaluate the TMG signal were determined: Td (delay time), Tc (contraction time), and Dm (displacement of muscle belly during contraction).
A typical TMG record, with all parameters defined, is shown in Figure \TODO{reference}.

In the measurements of rectus femoris muscle, the standard TMG protocol of increasing stimulation intensity was used [36][37][38].
The electrical current was increased after each twitch electrical stimulus (1ms duration) until the supramaximal muscle response was reached.
Once we have determined the current intensity in the first measurement in the supra-maximal response level, we used the same current intensity in all subsequent measurements before and after 4 “incline squat” (ISQ) sets.
We used the following potentiation protocol of exercises and measurements, repeated four times in a precise time sequence: pre-exercise measurement (PRE), after 15s, ISQ, after 12s, post-exercise measurement (POE), after 150s rest 2nd set and PRE, after 15s, ISQ, after 12s, post-exercise measurement (POE).
The complete procedure and timing of the exercise measurement protocol are depicted in Figure \TODO{reference}.
The volunteers performed ISQ on a platform with a slope angle of 30°, holding additional load (between 2 x 0 to 2 x 20kg in each hand (Figure \TODO{reference}a, \TODO{reference}b).
The load was chosen based on the 10- repetitions maximum test performed a day before.
The dynamic of the squat movement was 1s down, 1s up (a metronome kept the rhythm), and knee angles were between 0° and 90°.
The angle of the torso femur is constant throughout the movement at 0°.
Each subject performed four sets of ISQ, with eight repetitions in each set.

The effect of incline squat on the rectus femoris muscle was measured within the time frame of 12s after the last repetition in each set.
The rest between the two sets was 150s.
A pre-exercise measurement preceded the second, third and fourth sets.
15s after the measurement, the subjects started the next set of ISQ.

To estimate the load on the quadriceps muscles (especially RF, which is loaded in this exercise at the full amplitude of knee movement) during ISQ, an inverse dynamics analysis of a three-segment planar human body model was performed (see Appendix 1) [10][7][8], where quasi-static condition was assumed.
The body segment properties were estimated based on anthropometric data and the subject’s body height and body mass as input parameters [9].
The motion of the volunteers in sagittal plane was recorded with a video, where one repetition form the exercise was selected to provide the estimation of the inter-segmental angles (Figure 4).
The peak value of the knee joint net moment, occurring at the lowest body position during the workout routine, was estimated.
The knee joint moment was normalized to the volunteer’s body mass to reduce the effect of the inter-subject variance [5][6].
The normalized peak values of the knee joint moment were used to compare the workout loading of the knee extensor muscles between the volunteers.

\subsection{Statistical Analyses}
The statistical tests were made using SPSS Statistics 20 (SPSS Inc., Chicago, IL, USA), Microsoft Excel, and SPM1d t-statistics based on Statistical Parametric Mapping (SPM).
Statistical parametric mapping applies Random Field Theory to make inferences about the topological features of statistical processes that are continuous functions of space or time.
All SPM 1d t analyses were implemented in Python 2.7 (Van Rossum, 2014) using Canopy 1.6 (Enthought Inc., Austin, TX, USA) and spm1d [34][35] (http://www.spm1d.org).

Descriptive statistics were calculated on each variable, and the Shapiro–Wilk test was used to verify the normality of the data.
Data are presented as means ± standard deviations (SD).
The differences between pre-exercise and post-exercise conditions were determined using the Student’s dependent t-test.
All analyses were carried out with the significance level set at $ p \leq 0.05 $.

One dimensional paired t-test SPM statistics were performed to determine the main effects of ISQ on twitch-contraction properties/signal of RF muscle, the p-value was calculated for clusters crossing the critical threshold (t*), with significance set at p<0.05 (Pataky, 2013 [13]).
The significance level $ \alpha = 0.05 $ was set for all analyses, and specific p-values were reported for each significant region of the curve analyzed. In all SPM analyses, the first 100ms of RF twitch contraction signal was the region of interest.
One-dimensional SPM analysis returns regions of significance in the form of clusters (Pataky).
These are contiguous values over which the curve is determined to be not consistent with random sampling.
The following variables of the SPM statistics were collected: diff-set is the difference between pre- and post-exercise (ISQ) twitch contraction measured on rectus femoris muscle;
Centroid time is the time of the centroid of area Above threshold (see Figure 5a-d);
Centroid t-value is the position of the SPM t-axis value of area Above Threshold centroid;
Maximum is the t maximum value;
Area Above Threshold is the area (integral) of statistically significant difference between the mean of pre- and post-exercise twitch contraction applying SPM t-statistical analysis and corresponds to the level of potentiation;
Start Time is beginning of the statistically significant difference between pre- and post-exercise mean of rectus femoris twitch contraction;
End Time is the end of the statistically significant difference between pre and post-exercise mean of rectus femoris twitch contraction.

\section{Results}
For all four sets was p-value <0.0001 and $ \alpha = 0.05 $.
As an estimate of muscle load, the mean calculated value of the knee peak torque in ISQ was 126.5±17.6 Nm.
The comparison (Student's t-test) between the TMG variables (Td, Tc, Dm, RDD max, and RDDmaxt) of twitch contraction before and after the ISQ were statistically significantly different, and the p-value was<0.0001.
In the discrete analysis, we made two comparisons, the first of the muscle twitch before and after each set and the second comparison between the first measurement before and all 4 TMG measurements after the incline squat.
All the results of comparing variables and differences (\%) of muscle twitch before and after ISQ can be seen in more detail in Table 1 and Table 2.

% TODO: table 1
% TODO: table 2

SPM analysis (SPM t-test) showed a statistically significant difference in TMG measured muscle twitch before and after ISQ in all four sets (p-value<0.0001).
For all TMG signals, the region of interest was the first 100 ms of muscle twitch because this was the region in which the maximal amplitude of twitch contraction and the phase of twitch amplitude decrease occur (Figure 5).
The threshold t* values for the significance of the changes in all four measurement sets were constant between 2.76 and 2.79 SPM t.
The onset of significant difference in muscle twitch, start time (St) of RF, was at St=12.45ms in the first set, St=12.71ms in the second set, St=12.16ms in the third set, and St=11.97ms in the fourth set.
The end of the statistically significant difference (Et) was Et=59.8ms for the first set, Et=59.66ms for the second set, Et=58.1ms for the third set, and Et=58.82ms for the fourth set.
The maximum SPM t value (Stmax) in the first set was Stmax=7.77, in the second set Stmax=9.49in the third set Stmax=8.27and in the fourth set Stmax=8.12ms.
The Area Above Threshold (AAT) representing the magnitude of the probability of change in the first set was ATT=194.27, in the second set ATT=258.49, in the third set ATT=197.43, and in the fourth set ATT=208.84.
The other variables of the SPM t analysis and a detailed display of all variables can be seen in Table \TODO{reference}.

% TODO: table 3

\section{Discussion}

\section*{Conflict of Interest Statement}
The authors declare that the research was conducted in the absence of any commercial or financial relationships that could be construed as a potential conflict of interest.

\section*{Author Contributions}
The Author Contributions statement must describe the contributions of individual authors referred to by their initials and, in doing so, all authors agree to be accountable for the content of the work.
Please see \href{https://www.frontiersin.org/about/policies-and-publication-ethics#AuthorshipAuthorResponsibilities}{here} for full authorship criteria.

\section*{Funding}
Details of all funding sources should be provided, including grant numbers if applicable.
Please ensure to add all necessary funding information, as after publication this is no longer possible.

\section*{Acknowledgments}
This is a short text to acknowledge the contributions of specific colleagues, institutions, or agencies that aided the efforts of the authors.

\section*{Supplemental Data}
 \href{http://home.frontiersin.org/about/author-guidelines#SupplementaryMaterial}{Supplementary Material} should be uploaded separately on submission, if there are Supplementary Figures, please include the caption in the same file as the figure. LaTeX Supplementary Material templates can be found in the Frontiers LaTeX folder.

\bibliographystyle{Frontiers-Harvard}
\bibliography{test}
%%% Please see the test.bib file for some examples of references

\end{document}
