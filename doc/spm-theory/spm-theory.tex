\input{~/dotfiles/config/latex/templates/preamble}
\input{~/dotfiles/config/latex/templates/vector-calc-macros}

\begin{document}
    \textbf{What's involved}
    \begin{itemize}
    
        \item A set of $ I $ 1D (generally time-series) curves representing independent experimental observations.
        Let $ K $ denote the number of nodes (i.e. data points) per curve.

        Assemble curves into an $ I \times K $ matrix $ \mat{Y} $.

        Result: matrix $ \mat{Y} \in \mathbb{R}^{I \times K} $ where $ I $ is the number of curves and $ K $ is the number of data points per curve.

        \item A number $ J $ of experimental factors (generally two, e.g. for pre- and post-exercise)

        \item A matrix $ \mat{X} \in \mathbb{R}^{I \times J} $

        \item Describe a general linear parameteric statistical test using a general linear model for $ \mat{Y} $ of the form
        \begin{equation*}
            \mat{Y} = \mat{X} \vec{\beta} + \vec{\epsilon},
        \end{equation*}
        where $ \vec{\beta} \in \mathbb{R}^{J \times K} $ is a matrix of unknown regression parameters and $ \vec{\epsilon} \in \mat{R}^{I \times K} $ matrix of residuals.

        \item Least-squares estimates of the regression parameters are given by the pseudo-inverse
        \begin{equation*}
            \hat{\vec{\beta}} = (\mat{X}^{\top} \mat{X})^{-1} \mat{X}^{\top} \mat{Y}.
        \end{equation*}
        Interpretation: the parameter matrix $ \hat{\vec{\beta}} $ contains $ J $ curves reflecting the mean trends for each of the $ J $ experimental factors.
        
        \item Compute residuals associated with the least-squares estimate $ \hat{\vec{\beta}} $ via
        \begin{equation*}
            \vec{\epsilon} = \mat{Y} - \mat{X} \hat{\vec{\beta}}.
        \end{equation*}
        
        \item Compute variance curves $ \sigma^{2} $ via
        \begin{equation*}
            \hat{\sigma}^{2} = \frac{\mathrm{diag}(\vec{\epsilon}^{\top} \vec{\epsilon})}{I - \rank (\mat{X})},
        \end{equation*}
        which produces $ \hat{\sigma}^{2} \in \mathbb{R}^{K} $, i.e. the number of nodes in the curves in $ \mat{Y} $.
        
        \item The $ t $ statistic is computed according to
        \begin{equation*}
            t_{k} = \frac{\vec{c}^{\top} \hat{\vec{\beta}}_{k}}{\hat{\sigma}_{k} \sqrt{\vec{c}^{\top} (\mat{X}^{\top} \mat{X})^{-1} \vec{c}}},
        \end{equation*}
        where $ \vec{c} \in \mathbb{R}^{J} $ is a contrast vector assigning weights to the $ J $ experimental factors.
        A two-sample t test uses $ \vec{c} = (-1, 1)^{\top} $.
        
    
    \end{itemize}
\end{document}
